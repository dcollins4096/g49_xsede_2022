%\documentclass[11pt]{article}  % for e-submission to ApJ
\documentclass[11pt]{NSF}  % for e-submission to ApJ

%\documentclass[12pt,preprint2]{aastex}  % for e-submission to ApJ - two column

%\documentclass[manuscript]{emulateapj}  % this makes everything look like ApJ

\usepackage{graphicx, natbib, color, bm, amsmath, epsfig}
\input{aas_papers.tex}
\input{commands.tex}

\def\SUtotal{4.3\sci{4}       }                                                                       
\def\tfinal{T_{\rm{final}}}
\def\suzu{\ensuremath{\rm{SU}_{\rm{zu}}}}
\def\nzones{N_{\rm{z},\ell}}
\newcommand{\Lund}{\ensuremath{\mathrm{S}}}
%\newcommand{\SUestimate}[1]{\textcolor{red}{#1}}
\newcommand{\SUestimate}[1]{#1}
%\def\SUtotal{1.6\sci{5}       }                                                       
%\def\SUturb{1.8\sci{4}}
%\def\SUcore{1.2\sci{4}}
%\def\SUcmb{2.6\sci{4}}
%\def\SUgal{2.2\sci{3}}
%\def\DiskTurb{5.6\sci{3} Gb}
%\def\DiskCore{1.6\sci{5} Gb}
%\def\Diskcmb{1.7\sci{4} Gb}
%\def\Diskgal{1.1\sci{4} Gb}
%\def\StoreTotal{1.9\sci{5} Gb}
\def\nameTurbulence{\emph{turbulence}}
\def\nameTurbShort{\emph{turb}}
\def\nameCores{\emph{cores}}
\def\nameCMB{\emph{foregrounds}}
\def\nameGalaxies{\emph{galaxies}}
\def\suPerZoneUpTurbulence{\ensuremath{2.0\sci{-11}}}
\def\suPerZoneUpCores{\ensuremath{6.2\sci{-11}}}
\def\suPerZoneUpCMB{\ensuremath{6.2\sci{-11}}}
\def\suPerZoneUpGalaxies{\ensuremath{3\sci{-10}}}

\citestyle{aa}  % correct formatting for ApJ style files

\usepackage{aas_macros}
\begin{document}

\begin{centering}
\begin{LARGE}
Progress Report for

TG-AST140008

\end{LARGE}
\end{centering}


\pagestyle{plain}

During our 2020-2021 allocation we have primarily work on analysis of results and writing papers.
Four papers
from the \nameCores\ project are in preparation.
One paper for the
\nameSupernova\ project has been published as \citep{Hristov21}.

The first paper in the \nameCores\ project examines the initial conditions and
collapse rate of star forming clouds.  This paper will be submitted in April
of 2022. 
  We find that the cores do not always have a
phase where the velocity is subsonic, as was originally expected; and the
collapse is slower than free-fall in all but the largest objects, which are
substantially faster.

The second paper, to be submitted  by the Summer of 2022, is an examination
of a new analytical formulation of the distribution of energy in isothermal
turbulence.  This study needs a few more simulations to ensure the trends happen at
higher Mach numbers, but is otherwise nearing completion.

The third paper, to be submitted by Summer 2022,
examines the behavior of magnetic fields during the collapse.   It is found that
the ratio of magnetic field strength to density decreases as a function of time
by the act of turbulence alone.  

The fourth publication, to be completed by August 2022, is an examination of the
full suite of forces acting on collapsing gas.  

We concluded the \nameSupernova\ project this year, with the results published
in \citep{Hristov21}.  We find the surprising result that many Type Ia spernovae
require very large ($10^6$G), magnetic fields to reproduce the late-time light
curves observed.  

\bibliographystyle{apj}
\bibliography{apj-jour,Progress.bib}  % looks in ms.bib for bibliography info

\end{document}  


