The galactic magnetic field can be seen in Figure \ref{fig.planck}.  This figure
shows an all-sky projection of dust in the galaxy, as seen by the 353 GHz camera
on Planck.  The image is smeared along the direction of
the magnetic field, which is measured by way of the polarization of the signal.  Our goal in the \nameGalaxies\ project is to understand the
origin of this magnetic field.   We will We will discuss the background in Section
\ref{subsec.galaxies_motivate}, and describe the simulations we will perform in
Section \ref{subsec.galaxies_sims}

\subsubsection{Motivation: \nameGalaxies}
\label{subsec.galaxies_motivate}

The Milky Way has a large scale magnetic
field of roughly $\sim 5 \mu$G, about 200,000 time weaker than a refrigerator
magnet, but spanning the entire galaxy.   In the previous project, \nameCMB, the
goal is to study the properties of the field with high resolution, while in
the \nameGalaxies\ project the goal is to study the origin of this field.

The origin of this magnetic field is an open question.  There are presently two
known \emph{dynamos}, that is mechanisms to amplify magnetic fields. They differ in
two ways; the length scales over which they act, and the time scales over which
they act.  The fast
dynamo converts turbulent kinetic energy to magnetic energy at small scales, and
produces disordered fields quickly.  The slow dynamo produces large scale fields
slowly, with large scale convective motions. The magnetic field in the Milky Way, as well as other similar galaxies,
shows large scale order, but based on observations of old galaxies, must have been
built up quickly. 

The proposed simulations will simulate Milky Way sized galaxies and measure the
amplification of the magnetic field and process responsible for its amplification, and compare its
statistical properties of Galactic fields to those found in the \nameCMB\
project.

\subsubsection{Simulations: \nameGalaxies}
\label{subsec.galaxies_sims}

The disk of our simulated galaxies will be 500pc thick and 25kpc in radius.  Our
proposed simulation domain will begin at very large scale, $1.3$ Mpc.  This is to
separate the boundary from the region of interest, and to give gas expelled from
the galaxies
enough volume to expand.  This will begin at $256^3$, smaller than the
other simulations, but this suite of galaxy simulations has much deeper AMR.  We will resolve a nest of refinement grids, each one
1/2 of its parent grid on a side, giving constant number of zones per level.
This will be done for 5 levels.  We will allow the simulation to 
refine for a further 3 levels, based on the local density of the gas.  Eight
levels then gives us 20pc of resolution on the finest
level, so we will resolve molecular clouds by a few zones.  We will have ample
resolution in the disk to study the dynamo action as it occurs, and sufficient
resolution in the CGM to serve as an appropriate boundary.  As we are simulating
the entire galaxy, we can no longer use an idealized isothermal equation of
state as the other simulations do, but will use ISM heating and cooling functions by way of the
tabulated look up using Grackle \citep{Smith17}.  We will perform two such
simulations, one production simulation and one for development.
Simulations will last for 1Gyr, four orbital timescales for the galaxy. 

These simulations will also be useful in conjunction with the \nameCMB\ project.
The two approaches compliment each other, as the \nameCMB\ simulations will
resolve the turbulence with great detail, but the \nameGalaxies\ simulations
will capture the multiphase nature of the ISM and the large scale morphology.
