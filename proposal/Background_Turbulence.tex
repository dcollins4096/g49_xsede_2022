Turbulence underpins all of the projects in this proposal, the first two most
notably.  Here we describe the aspects of a turbulent system that impact the
design of our simulations, the excellent work of \citet{Frisch95} has more
details.

In a turbulent system, energy is injected at a scale, $L_{driving}$.  By way of
fluid instabilities, the energy cascades to smaller and smaller scales until
molecular interactions dissipate the energy at a scale $L_{diss}$.  The energy
spectrum 
behaves like a powerlaw, $E(k)=k^{a}$.  For incompressible fluid turbulence,
$a=-5/3$, but that exponent varies with the inclusion of supersonic
(compressible) flow and magnetic fields.  The region between
$L_{driving}$ and $L_{diss}$, where the powerlaw is valid, is referred to as
the \emph{inertial range}.  Here, the nonlinear inertial terms in the
Navier-Stokes equation dominates over the dissipation terms, and it is the
behavior of the nonlinear terms that we wish to explore.  $L_{driving}$ is set
by the boundary of the simulation. $L_{diss}$ is set by details of the
solver and the energy of the flow.  Thus, for a given driving energy, the only
way to increase the size of the inertial range is with resolution.  The
inclusion of magnetic fields only shrinks the inertial range and makes the
problem harder.

At a resolution of $256^3$, the driving scale and dissipation scale overlap, and
there is not much inertial range to work with.  At $512^3$, an inertial range
appears, but it is small, and determining the actual powerlaw portion of the
spectrum is error-prone.  This will be demonstrated in Section
\ref{sec.back_foregrounds}.  At $1024^3$, a reasonable powerlaw appears even for
magnetized simulations.  Further increase of the resolution is desirable, but
the cost scales like the number of zones on a side to the fourth power, while
the size of the inertial range is only linear.  Increases in resolution to
$2048^3$ and beyond is possible and desirable, but also quite costly in terms of
total $SU$, disk, and human time.  This will be explored if the results of the
current proposal indicate it is necessary.  

Turbulence is a chaotic process, and must be handled statistically.  A single
snapshot from a turbulent box is largely random, the true nature must be
averaged over a window of time.  Additionally, we focus on fully developed
turbulence, which takes some simulation time to develop. Thus we run our
simulations for a period of time to develop the turbulence, and a period of time
for statistical averaging.  This is discussed in detail in the next two
sections.

%For the \nameCMB\ and \nameTurbulence\ simulations, we will use a resolution of $1024^3$ to
%obtain inertial ranges that are well defined.  For the \nameCores\ suite, we
%will use a root grid of $512^3$ to offset the cost of 4 levels of AMR and
%gravity.  The
%\nameGalaxies\ will use a root grid of $256^3$ and 8 levels of AMR, and
%sacrifices well resolved turbulence for spatial dynamic range.  Properly
%resolving the turbulence in the galaxy as well as the circumgalactic media is
%presently impossible.  

Simulation purpose and design will be described further in the next
sections.
