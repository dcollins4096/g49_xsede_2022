\section{Simulation Plan}
\label{sec.plan}

\begin{table} \begin{center} \caption{Allocation request.  Cost for each simulation, $SU$, is computed from
wall time and number of nodes for each suite, per Equations \ref{eqn.twall},
\ref{eqn.su}.  $T_{wall}$ is measured in hours, the number of nodes for each
suite can be found
in Table \ref{table1}.
The \nameTurbulence\ and \nameCMB\ suites are
itemized by Mach number and \alf\ Mach number, $M_s$ and $M_a$,
which affect the total time, T, and timestep size $\Delta T$.  
The AMR simulations, \nameCores\ and \nameGalaxies\, is are itemized for a
single simulation by level, $\ell$, and the cost is found by estimating the
volume fraction, $f_\ell$, covered on level and the
time step $\Delta t$ for that level.  The \nameCores\ and \nameGalaxies\
simulations are repeated 3 and 2 times.  Long term disk usage is estimated as $N_Z$
    times the number of fields for each simulation.  More details are given in
    the text. }
 \label{table2}                                                                                                                                                               
\begin{tabular}{l               c               r               r               r                       r                       r               r               r       }       
   suite       &$M_s, M_A$       &               &     \Nz       &       T               &$\Delta T$               &     \Nu       &$T_{wall}$       &      SU             \\
  \hline                                                                                                                                                               
\nameCMB       &     1,1       &               &1.1\sci{9}       &     2.5               &4.3\sci{-5}               &5.8\sci{4}       &    42.0       &2.7\sci{3}             \\
\nameCMB       &     1,5       &               &1.1\sci{9}       &     2.5               &1.4\sci{-5}               &1.7\sci{5}       &   126.1       &8.1\sci{3}             \\
\nameCMB       &     5,1       &               &1.1\sci{9}       &     0.5               &1.4\sci{-5}               &3.5\sci{4}       &    25.2       &1.6\sci{3}             \\
\nameCMB       &     5,5       &               &1.1\sci{9}       &     0.5               &8.7\sci{-6}               &5.8\sci{4}       &    42.0       &2.7\sci{3}             \\
  \hline                                                                                                                                                               
               &               &               &               &                       &                       &               &      SU       &1.5\sci{4}             \\
               &               &               &               &                       &                       &               &    Disk       &9.0\sci{3}             \\
   suite       &   $M_s$       &               &     \Nz       &       T               &$\Delta T$               &     \Nu       &$T_{wall}$       &      SU             \\
  \hline                                                                                                                                                               
\nameTurbulence       &     0.5       &               &1.1\sci{9}       &       5               &2.8\sci{-5}               &1.8\sci{5}       &   128.3       &8.2\sci{3}             \\
\nameTurbulence       &       1       &               &1.1\sci{9}       &     2.5               &2.1\sci{-5}               &1.2\sci{5}       &    85.5       &5.5\sci{3}             \\
\nameTurbulence       &       2       &               &1.1\sci{9}       &    1.25               &1.4\sci{-5}               &8.8\sci{4}       &    64.2       &4.1\sci{3}             \\
\nameTurbulence       &       4       &               &1.1\sci{9}       &   0.625               &8.5\sci{-6}               &7.3\sci{4}       &    53.5       &3.4\sci{3}             \\
\nameTurbulence       &       7       &               &1.1\sci{9}       &   0.357               &5.3\sci{-6}               &6.7\sci{4}       &    48.9       &3.1\sci{3}             \\
\hline
               &               &               &               &                       &                       &               &      SU       &2.4\sci{4}             \\
               &               &               &               &                       &                       &               &    Disk       &4.0\sci{3}             \\
   suite       &  $\ell$       &$f_\ell$       &     \Nz       & T [Myr]               &$\Delta T$               &     \Nu       &$T_{wall}$       &      SU             \\
  \hline                                                                                                                                                               
\nameCores       &       0       &       1       &1.3\sci{8}       &       1               &4.6\sci{-3}               &2.2\sci{2}       &     0.2       &1.7\sci{0}             \\
\nameCores       &       1       &4.6\sci{-1}       &4.9\sci{8}       &       1               &2.3\sci{-3}               &4.4\sci{2}       &     1.6       &1.3\sci{1}             \\
\nameCores       &       2       &8.3\sci{-2}       &7.1\sci{8}       &       1               &1.1\sci{-3}               &8.7\sci{2}       &     4.6       &3.7\sci{1}             \\
\nameCores       &       3       &1.3\sci{-2}       &8.7\sci{8}       &       1               &5.7\sci{-4}               &1.7\sci{3}       &    11.2       &9.0\sci{1}             \\
\nameCores       &       4       &1.8\sci{-3}       &1.0\sci{9}       &       1               &2.9\sci{-4}               &3.5\sci{3}       &    25.7       &2.1\sci{2}             \\
  \hline                                                                                                                                                               
               &               &               &               &                       &                       &               & per sim       &3.5\sci{2}             \\
               &               &               &               &                       &                       &               &      SU       &1.0\sci{3}             \\
               &               &               &               &                       &                       &               &    Disk       &2.0\sci{4}             \\
   suite       &  $\ell$       &$f_\ell$       &     \Nz       & T [Gyr]               &$\Delta T$               &     \Nu       &$T_{wall}$       &      SU             \\
  \hline                                                                                                                                                               
\nameGalaxies       &       0       &       1       &1.7\sci{7}       &       1               &3.5\sci{-4}               &2.8\sci{3}       &     0.4       &2.8\sci{0}             \\
\nameGalaxies       &       1       &1.3\sci{-1}       &1.7\sci{7}       &       1               &1.8\sci{-4}               &5.7\sci{3}       &     0.7       &5.6\sci{0}             \\
\nameGalaxies       &       2       &1.6\sci{-2}       &1.7\sci{7}       &       1               &8.8\sci{-5}               &1.1\sci{4}       &     1.4       &1.1\sci{1}             \\
\nameGalaxies       &       3       &2.0\sci{-3}       &1.7\sci{7}       &       1               &4.4\sci{-5}               &2.3\sci{4}       &     2.8       &2.2\sci{1}             \\
\nameGalaxies       &       4       &2.4\sci{-4}       &1.7\sci{7}       &       1               &2.2\sci{-5}               &4.5\sci{4}       &     5.6       &4.5\sci{1}             \\
\nameGalaxies       &       5       &3.1\sci{-5}       &1.7\sci{7}       &       1               &1.1\sci{-5}               &9.1\sci{4}       &    11.2       &9.0\sci{1}             \\
\nameGalaxies       &       6       &3.8\sci{-6}       &1.7\sci{7}       &       1               &5.5\sci{-6}               &1.8\sci{5}       &    22.4       &1.8\sci{2}             \\
\nameGalaxies       &       7       &4.8\sci{-7}       &1.7\sci{7}       &       1               &2.8\sci{-6}               &3.6\sci{5}       &    44.9       &3.6\sci{2}             \\
\nameGalaxies       &       8       &6.0\sci{-8}       &1.7\sci{7}       &       1               &1.4\sci{-6}               &7.3\sci{5}       &    89.7       &7.2\sci{2}             \\
  \hline                                                                                                                                                               
               &               &               &               &                       &                       &               &SU per sim       &1.4\sci{3}             \\
               &               &               &               &                       &                       &               &      SU       &2.9\sci{3}             \\
               &               &               &               &                       &                       &               &    Disk       &1.1\sci{3}             \\
  \hline                                                                                                                                                               
  \hline                                                                                                                                                               
               &               &               &               &                       &                       &               &      SU       &4.3\sci{4}             \\
               &               &               &               &                       &                       &               &    Disk       &3.4\sci{4}             \\
\end{tabular}                                                                                                                                                               
\end{center}                                                                                                                                                               
\end{table}                                                                                                                                                                


Table \ref{table2} shows the itemized cost for each suite of simulations, which
we discuss in this section.

The wall time for these simulations is found as
\begin{align}
    T_{wall} &= \frac{N_Z N_U}{N_C \zeta} \frac{1}{3600}\label{eqn.twall}\\
    \zeta &= \frac{zone\ updates}{core\ second},
\end{align}
where $N_Z$ is the number of zones, $N_U$ is the number of updates, $N_C$ is the
total number of cores used, and $\zeta$ is the performance of the code given the
physics employed.  The net cost in $SU$ is then
\begin{align}
    SU = T_{wall} N_N, \label{eqn.su}
\end{align}
where $N_N$ is the number of nodes.  For all simulations, we will use 64 cores
per node, so $N_C=64 N_N$.  We will estimate $N_Z$ and $N_U$ from the physics
goals in this section.  The performance, $\zeta$, is measured in the Scaling
document.  The number of nodes, $N_N$, is selected from the performance
described in the scaling document.  It is found that the simulations without
self-gravity (\nameCMB\ and \nameTurbulence) scale quite well, and will use 64
nodes.  The two with gravity do not scale as well due to the gravity and AMR
overhead, and will use 8 nodes.

The estimate of the number of zones, $N_Z$,
is determined by the target resolution for the simulation and the expected AMR
structure.  For the fixed
resolution simulations, this is trivial.  
For the AMR
simulations, the actual number of zones  
 is dynamically determined by the portion of the flow
that is turning into stars.  
This is a chaotic process, so formally impossible
to predict.
However, it can be expected to be roughly similar to
previous simulations, so we estimate the covering fraction from those.   
We measure the covering fraction, $f_\ell$, of AMR grids on each level, $\ell$,
from prior simulations, and compute the number of zones on each level as
\begin{align}
    N_Z = \frac{V f_\ell}{\Delta x_\ell^3},
\end{align}
where $V$ is the total volume for each simulation, and $\Delta x_\ell$ on each
level is 1/2 that of its parent.

The number of updates, $N_U$, is found as
$N_U=T/\Delta T$, where the total simulation time is $T$ and the size of the
timestep is $\Delta T$.  $T$ is determined by the physics
problem.  The size of the time step $\Delta T$ is
determined by a standard Courant condition, that is a wave cannot cross half of
one zone in a timestep, 
\begin{align}
\Delta T = \eta \frac{\Delta
x}{\vsignal}, \label{eqn.cfl}
\end{align}
and the safety factor $\eta = 0.5$.  We determine $v_{\rm{signal}}=c_s+v_{\rm{max}}$
 as the sum of the sound speed and the max velocity, from preliminary studies, and then use use Equation \ref{eqn.cfl} to
determine the number of steps on each level.

Both the \nameTurbulence\ and \nameCMB\ simulations are fixed resolution and
employ only the random forcing and hydro/MHD solver.    Both will be
run at $1024^3$.  The total time, $T$, is 5 shock-crossing times, so $T =
5 L/\mach$, where $L$ is the size of the driving pattern.  The
timestep, $\Delta T$, also decreases with Mach number as Equation \ref{eqn.cfl},
and is determined measuring the signal speed \vsignal from
previous fully developed turbulence simulations and rescaling with the Mach
number.

The \nameCores\ simulations will have $512^3$ root grid zones and $1024^3$
particles, as well as 4 levels of AMR.  The refinement will be based on the
density.  It will use the isothermal MHD solver, the gravity solver, and the
particle update machinery.  The net cost per zone update for this combination of
physics solvers and a similar AMR structure to the production simulations is
discussed in the Scaling document.  We will perform three of these simulations.

The \nameGalaxies\ suite will restrict the dynamic AMR to the disk of the galaxy, and
use a tower of refinement, with each level 1/2 the length of the parent, to separate the outer
part of the CGM at 1.3 Mpc from the small star forming regions in the disk.  The
first 5 levels will be static nested AMR levels, the final 3 will be dynamic.
For each level we approximate that about 10\% coverage of the parent level.
This is by construction for the first 5, and from experience with similar
simulations for the final 3.  These simulations will use the MHD solver, the
gravity solver, the particle machinery for the star particles.  We have performed a preliminary
simulation using 5 levels to determine the 
anticipated signal speed, \vsignal, to determine the timestep size. 

Table \ref{table2} shows the breakdown of the total request by simulation.  The
\nameTurbulence\ and \nameCMB\ suites are itemized by Mach number, while the
\nameCores\ and \nameGalaxies\ suites are itemized by AMR level.  Shown in that
table is the name; the parameter, either Mach number, level, or Sonic and \alf\
Mach numbers;  the volume fraction; the number of zones $N_Z$; the total
simulation time $T$; the timestep size $\Delta T$; the total number of
updates $N_U$; the wall time $T_{wall}$ in hours; and
finally the total SU cost.  The table also displays disk usage for these
simulations.  The \nameCores\ suite will be repeated 3 times, and the
\nameGalaxies\ will be repeated twice.

The {\bf long term disk storage} requested has two portions; the new
simulations, and our archive of simulations that are still bearing fruit.
Our existing archive of previous simulations is $7\sci{4}$Gb.
Disk usage for the current request, presented in Table 2, is estimated from the number of zones, $N_Z$.  Each zone
stores a number of fields, $N_F$: 5 for \nameTurbulence (density, 3 velocity, and
energy); 14 for \nameCores\ and
\nameCMB (density, energy, 3 velocity, 3 magnetic fields, 3 electric fields, 3
additional magnetic fields, see \citet{Collins10}); 24 for the \nameGalaxies\
suite (14 for MHD, 10 for additional chemistry fields.)  So the total memory is
8 bytes for all of $N_Z\times N_F$ fields.  It is listed in Gb in the table.

