Turbulence is ubiquitous in astrophysical processes.  We have developed analytic
predictions for the distribution of energies in isothermal turbulence, both
supersonic and subsonic. These simulations will verify our analytic predictions,
and examine if deviations seen in preliminary studies are due to lack of
resolution or interesting physics.
We provide the background for the \nameTurbulence\ project in Section
\ref{subsec.turb_motivate},
and motivate the simulations that support this study in Section
\ref{subsec.turb_sims}.


\subsubsection{Motivation: \nameTurbulence}
\label{subsec.turb_motivate}

The interstellar medium (ISM) is the gas between stars in the galaxy.  It cools
very effectively, so can be treated as isothermal \citep{Krumholz14}.  The ISM is also turbulent, with supersonic shocks driven by supernovae
causing supersonic turbulence throughout the interstellar medium
\citep{Elmegreen04}.  This
turbulence impacts the formation of stars (see Section \ref{sec.back_cores}) and
causes a polarized screen that is blocking our view of the light from the big bang (see
Section \ref{sec.back_foregrounds}), among many other effects
\citep{Elmegreen04}. It is also interesting in its own right.  We have developed
analytic formulae for the probability density function (PDF) of internal energy
and kinetic energy, as well as their joint distribution.  In this project we
will verify these formulae with high resolution simulations.

Supersonic turbulence is compressible, and the distribution of density
fluctuations is described by a log normal, i.e. the log of density is
distributed as a Gaussian \citep{Vazquez-Semadeni94}.  The distribution of velocity is roughly Maxwellian,
i.e. each of the three velocity components is a Gaussian, and added in quadrature the
distribution is Maxwellian.  
We have recently found analytic distributions for the internal energy and
kinetic energy, as well as their joint
distribution.   (Rabatin et al 2022, in prep). Kinetic energy is defined in the familiar way, $E_K=\half \rho
v^2$.  Internal energy is defined as $E_T= c_s^2 \rho \ln \rho/\rho_0$, where
$c_s$ is the sound speed and $\rho_0$ is the mean density \citep{Banerjee18}.  The joint
distribution of these two quantities can be seen in Figure \ref{fig.energy},
along with our analytic prediction (dotted lines).
Each panel shows the joint PDF of $E_K$ and $E_T$;  the color field shows the
PDF derived from data; the solid lines show logarithmically spaced contours of
the data; the dashed lines show the theoretical prediction.  The first panel is
subsonic, with $\mach=0.5$, the middle has $\mach=1$, and the third panel is
slightly supersonic, with $\mach=2$.  Significant changes in the behavior of the
distribution, with high $E_K$ gas developing along side high $E_T$ gas, are both
predicted by the analytical dashed lines and seen in the data.  This series of
figures shows the difficulty of getting gas to large densities, i.e. large
values of $E_T$.  Gas with low $E_K$ is found predominantly with low $E_T$,
while at higher kinetic energy the thermal energy distribution becomes more
broad, creating higher density gas.  This will be useful in understanding, e.g.,
what gas in a molecular cloud becomes dense enough to form stars.

Our preliminary simulations were run with a modest resolution of $256^3$.  This
is enough to find reasonable agreement, but imperfect. This can be seen most
easily in the first panel of Figure \ref{fig.energy}, by examining the center
most (red) contours. The solid line shows simulation, while the dashed line
shows theory, which clearly agrees, but only approximately.  This lack of
agreement can be one of several things, the first to examine is numerical
resolution.  As discussed in Section \ref{sec.back_foregrounds}, at low
resolution the driving range and dissipation ranges intersect.  This causes
results that are determined by the numerics, not the nonlinear dynamics of the
equations.  
With insufficient resolution, energy is transferred from large to
small scales faster than would be natural, which could be why we have mediocre
agreement in our predicted theory.  

\subsubsection{Simulations: \nameTurbulence}
\label{subsec.turb_sims}

The proposed simulations will quadruple this
resolution to $1024^3$.  As discussed in Section \ref{sec.turbulence},
this resolution is needed to separate the driving and dissipation rages.  
Additionally, the resolution dictates the peak values of density (and other
conserved quantities such as $E_K$ and $E_T$) that the system can attain, as a
high density parcel of gas is necessarily small by 
mass conservation.  A resolution of $1/1024^3$ is necessary to resolve the
shocked gas that we will create.
If the
fits are not improved at $1024^3$, we will propose larger simulations during a
future call for proposals.

Turbulence is created in the same manner as the previous CMB simulations
\ref{sec.back_foregrounds}, wherein large scale energy is added at every time
step, and allowed to cascade to small scales.  Following the reasoning set out
in Section \ref{sec.cmb_sims}, we will run for 5 $t_{dyn}=L/V$, where $L$ is the
scale of the driving and $V$ is the r.m.s. Mach number.  This allows for $2
t_{dyn}$ for the establishment of the turbulence, and 3 statistically separate
periods to average the PDFs.  In both suites of simulations, 5 $t_{dyn}$ is the minimum
amount of time necessary to achieve meaningful results.

Owing to the substantial evolution with Mach number, we
will run some subsonic cases (\mach=0.5, 1.0) and some moderate and highly
supersonic cases (\mach=2,4,7).   These are dimensionless physics
studies, so the box size and sound speed are unity.
