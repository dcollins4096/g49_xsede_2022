\section{Introduction}
\label{sec.intro}

We are requesting \SUtotal\ SUs on Stampede 2 for the period beginning June 1,
2022.  This allocation will support four projects involving astrophysical
magnetic fields and turbulence.  The first project (\nameTurbulence) explores analytical
formulae we have developed for isothermal turbulence, which is relevant for many
astrophysical processes, among them the formation of stars.  The second project
(\nameCores) examines fractal structures in star forming clouds.
  The third project (\nameCMB) examines the polarized signal
produced by the interstellar medium, which is in the foreground of our understanding of
the distant cosmic microwave background (CMB). The fourth project (\nameGalaxies)
simulates entire galaxies, in order to understand the growth of the magnetic
field.
This research is supported by two NSF grants.  The first two projects
(\nameTurbulence\ and \nameCores)  
are
supported by NSF AST-1616026, and the third (\nameCMB) is supported by 
NSF AST-2009870.
We are hopeful that the \nameGalaxies\ project will be funded by a pending
proposal.

These projects support three graduate students.  Luz Jimenez Vela is working on
the \nameCores\ project; Branislav Rabatin is working on the \nameTurbulence\
and \nameCMB\ projects; and Jacob Strack is working on the \nameGalaxies\
project.


Table \ref{table1} shows the cost for each project.  Each of the four projects
uses a slightly different physics package, which affects  the cost of the
simulation.  In addition, two of the four projects employ adaptive mesh
refinement (AMR), a technique that adaptively changes the resolution of the
simulation.  This also affects the cost of the simulation.
We motivate each project and describe the simulations to be
run in Section
\ref{sec.background}.  In Section \ref{sec.method} we
describe the computational tools to be used.  In Section \ref{sec.plan} we
give the projected cost  and disk usage of these simulations.

