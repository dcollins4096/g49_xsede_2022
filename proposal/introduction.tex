\section{Introduction}
\label{sec.intro}

We are requesting \SUtotal\ SUs on Stampede 2 for the period beginning October 1,
2022.  This allocation will support four projects involving astrophysical
magnetic fields and turbulence.  The first project  (\nameCMB) examines the polarized signal
produced by the interstellar medium, which is in the foreground of our understanding of
the distant cosmic microwave background (CMB). The second project (\nameTurbulence) explores analytical
formulae we have developed for energy distributions in isothermal turbulence,
which are relevant for many
astrophysical processes, among them the formation of stars.
  The third project (\nameCores) examines fractal structures in star forming clouds. The fourth project (\nameGalaxies)
simulates galaxies in order to understand the growth of the magnetic
fields and their spatial distribution.  This research is supported by two NSF grants.  The first two projects
(\nameTurbulence\ and \nameCores)  
are
supported by NSF AST-1616026, and the third (\nameCMB) is supported by 
NSF AST-2009870.
We are hopeful that the \nameGalaxies\ project will be funded by a pending
proposal.

These projects support three graduate students.  Luz Jimenez Vela is working on
the \nameCores\ project; Branislav Rabatin is working on the \nameTurbulence\
and \nameCMB\ projects; and Jacob Strack is working on the \nameGalaxies\
project.


Table \ref{table1} shows the cost for each project in SUs, as well as the disk
usage, physics packages, number of simulations to be run, and number of nodes to
be used.  64 cores per node will be used for all runs.  Also included is the
storage need for our existing archive of data, much of which is still generating
publications.
Two projects are fixed resolution driven turbulence (\nameTurbulence\ and \nameCMB), and two use
more expensive gravity and adaptive mesh refinement (\nameCores\ and
\nameGalaxies). In Section \ref{sec.method} we
describe the computational tools to be used.  
We motivate each project and describe the simulations to be
run in Section
\ref{sec.background}.  In Section \ref{sec.plan} we
discuss the projected cost  and disk usage of these simulations.

