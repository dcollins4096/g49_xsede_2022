
The formation of stars is filled with filamentary, fractal structures
\citep{Andre14}.  The
\nameCores\ project will study the behavior of collapsing gas as it transitions
from fractal structures to form dense prestellar cores.  We will motivate the
study in Section \ref{subsec.cores_motivate}, and describe the simulations
in \ref{subsec.cores_sims}.


\subsubsection{Motivation: \nameCores}
\label{subsec.cores_motivate}
The formation of stars is one of the most important processes in astrophysics,
as stars provide most of the light we see in the night sky, and they produce the
energy and metal
that dictates the structure and composition of a galaxy.  
A \emph{prestellar core} is a knot of dense gas
formed by gravity in a molecular cloud, which will ultimately form a star.  The
formation of these cores is one of the more difficult puzzles in star formation,
as it is fundamentally dictated by chaotic dynamics of the turbulence in the
cloud.  This is the focus of the \nameCores\ project.

One of our previous studies (Collins et al 2022, in prep) examined the collapse of a molecular cloud by
including semi-Lagrangian tracer particles that follow the flow.  The particles
that are found in dense cores at the end of the simulation are then followed
backwards in time to 
to examine the \emph{preimage} of the gas, before it collapses.  This will allow
us to better constrain star formation models.  

One of the curious findings is the fractal nature of the preimage gas.  We find
that gas from distinct cores at the end of the simulation begins life mixed with
one another in a fractal manner. This can be seen in Figure
\ref{fig.cores}, which shows the overlap of a collection of preimages at the
beginning of the simulation in the first 3 panels, and the
collapse to form dense cores in the fourth.  The length scale in the first three panels translates to
roughly 2 pc in size.  These panels show the overlap and fractal, filamentary nature of
the gas before it collapses.  The
last panel shows the un-mixing of the gas as it collapses in time.

\subsubsection{Simulations: \nameCores}
\label{subsec.cores_sims}

The simulation presented in Figure \ref{fig.cores} was one of a suite of three
simulations with varied magnetic field strength.  We began with fully developed
turbulence, and then added tracer particles and turned on gravity and AMR.  These simulations were
relatively low resolution as we developed the analysis techniques and learned
about extended nature of the primage gas. Our proposed simulations will greatly improve the resolution
to explore if these structures are in fact fractal, or a result of low
resolution. 
The target simulations will have $512^3$ root grid and 4 levels of
AMR, and $1024^3$ particles.  This resolution strategy is a compromise between
a large root grid to resolve the turbulence and deep AMR to capture the cores.  

The size of the box is 5 pc, and the r.m.s. Mach number is 9.  They
will begin from existing simulation data.  

The proposed suite of three simulations will mirror the magnetic field strengths
of the preliminary studies.  The simulations will be similar to those presented
in \citep{Collins12}, but with tracer particles and sinks.
It will run for a total of one free-fall time, where $t_{ff}=(G
\rho)^{-1/2}=1$Myr (G is the gravitational constant and $\rho$ is the mean
density of the box).  Beyond this time, feedback from young stars begins and
alters the structure of the cloud, which is beyond the scope of this study. 

