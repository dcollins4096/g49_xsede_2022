Dear Dr. Collins:
The Research Research met on 2022-06-06 and considered your request. Your request was not recommended for an award.

Status: Declined

Allocations administrator comments: The recommendation of the XRAC panel is that your request be rejected, for reasons stated within the reviews. You should revise your proposal and submit the revised proposal in the next quarter. The submission window is 06/15/2022 - 07/15/2022. Please see the exemplary examples https://portal.xsede.org/allocations/research\#examples, as well as additional information Preparing Successful Request in the Allocations policy document, https://portal.xsede.org/allocations/policies Please request a time extension of your current award for an additional 3 months to be able to continue research then submit a supplement request if SUs are needed to prepare the revised proposal. Once a revised proposal is submitted an advance of SUs of up to 10\% of the pending request may be requested

The PI should submit a revised request in the next submission period (comments in the Request Summary section of XRAS will
indicate options available for a rejected request).

For question regarding this decision, please contact help@xsede.org.

Sincerely,

XSEDE Resource Allocation Service 

=============================================
REVIEWS
=============================================

Review \#0
Overall Rating: Reject

Assessment and Summary: This is a multi-project project that aims to investigate the validity of the analytic framework developed by the team for isothermal turbulence, which is relevant for star formation. Another project concerns the study of fractal structures in star forming clouds. The third project studies the polarized signal produced by the interstellar medium, and the last project simulates the growth of the magnetic field in a galaxy. This research is supported by two NSF grants with a pending third one and is supporting three graduate students who will execute the work.However there is a lack of justification on some runs and the Enzo code does not really scale well. The fourth project is not well justified neither. I recommend rejection.

Appropriateness of Methodology: the nested and zoomed simulations are not sufficiently justified

Appropriateness of Computational Research Plan: The scaling relations and performance tests are not clear

Review \#1
Overall Rating: Reject

Assessment and Summary: This renewal is a substantial increase from previous allocation. However the justification for this is lacking. For example for the "turbulent" and "foreground" requests SU for 4X increase of resolutions (from previous work) and will be run for 5 dynamical crossing time. Why 4X and why 5 crossing times? How do you know 4X resolution is enough or too much? Is it really necessary for 5X crossing time or why 2X crossing time is not enough? These are not justified in the proposal.  The SU request for "galaxies" also does not make sense. Based on the request it would take more than 1000 days for it to complete (i.e. 3 years). Large number of restarts will also be needed for the other simulations. I am not convinced the the computational plan has been thought through sufficiently. The scaling plot is confusing to me. How is it strong scaling with the cost increase (Y-axis) as the number of nodes increase (X-axis). It is very unclear what resolution is used for the scaling plot. For these issues I recommend rejection of this renewal.

Appropriateness of Methodology: The methodology is unclear.

Appropriateness of Computational Research Plan: Computational research plan is unclear.
