The cosmic microwave background (CMB) is the light leftover from the creation of
the universe.  It has taught us a considerable amount about the structure,
history, and future of the
universe.  To learn more from it, we must understand its polarization.  To see
the polarization of the CMB, we must first understand the polarization of the
interstellar medium (ISM), which is much brighter and in the way.  This project
will perform simulations of driven magnetized turbulence, and compare the
structures found on the sky.
We will
motivate this project in Section \ref{subsec.cmb_motivate}, and describe the
simulations we will perform in Section \ref{subsec.cmb_sims}

\subsubsection{Motivation: \nameCMB}
\label{subsec.cmb_motivate}

When the universe was very young, it was very small, and also very hot.  So hot
that everything was ionized.  Sound waves generated by the big bang travel through
the universe, causing temperature fluctuations.  Once the universe cooled
enough, the electrons and protons combined to make the first Hydrogen.  After
this time, photons can travel great distances without running into an electron.
These photons are the CMB.  It is an extremely uniform 2.7K black body.  Small
($\mu K$) fluctuations in this temperature have been studied extensively by
satellites such as Planck \citep{Planck2018VI20}. These fluctuations have answered many
questions about the content and eventual fate of the universe.  But there are
still open questions.

Why is the CMB a single temperature?  The
universe is very large, and one side of the observable universe has never been
in contact with the other.  Given the expansion of the universe, one would
expect significant fluctuations in temperature on the angular size of the full moon on the sky.
One possible answer is an extremely rapid
\emph{inflation} of the universe, where the universe expands from the size of a
proton to the size of the solar system within the first $10^{-16}$s.  This is
different from the \emph{expansion} of the universe, which has been well
established and a much more leisurely pace.  Such a
rapid event would allow different patches of the universe to be at the same
temperature very early on.  
It is also extraordinarily violent, would leave a sea of gravitational waves.  These gravitational
waves, being quadrupolar in nature, imprint a polarization on the CMB. To detect
this polarization is to witness the violent birth of the Universe.

Unfortunately (for observing the CMB) the Galaxy we live in is filled with dust,
which gives a polarized signal in the same frequency range as the polarized CMB.
This dust, which includes iron and magnesium, lines up perpendicular to the magnetic
field in the galaxy, not unlike iron filings around a bar magnet.  These aligned
grains radiate polarized thermal radiation in the microwave and infrared, and so
the dust polarization is perpendicular to the magnetic field.  This
polarized signal is much brighter than the polarization in the CMB, so it must be
removed.  In order to remove it, we must understand the statistical properties
of the turbulent interstellar medium of the Galaxy. 

The polarization vector of a photon is the given by orientation of its electric field.  This is
a quantity that depends on the orientation of the camera observing it, which is
difficult to use since the camera must point over the whole sky.  We define two
rotationally invariant quantities, $E-$mode and $B-$mode, which are the
parity-even and parity-odd versions of the polarization vector.  Briefly,
$E-$mode is polarization at either 90 or 45 degrees to filamentary structure,
and $B-$mode is polarization that is at oblique angles to filaments.  More
details can be found in \citet{Rotti19}.  

The Planck satellite \citep{PlanckXIX15} measured the galactic polarization, and the result can be
seen in Figure \ref{fig.planck}.  The left panel shows the 353 GHz dust emission in color,
and the image is smeared along the direction of the magnetic field at every
point.  The right panel shows the power spectrum of the polarization from
simulations (colored lines) and Planck (black line).   This plot show
$C_\ell^{EE}$, the amplitude in $E-$mode power, vs wavenumber $\ell$.
It is found \citep{PlanckXIX15} that both quantities are
distributed over all scales in a power-law fashion, with $E \propto
\ell^{-2.5}$.
$B$ has a similar exponent but half the
amplitude.  

The colored lines in Figure \ref{fig.planck} are the results of a suite of MHD
simulations to be published in Stalpes et al 2022 (in prep).  This was a suite
of magnetized turbulence simulations at various r.m.s velocities and magnetic
field strengths at a resolution of $512^3$ performed on \emph{Stampede2}.  The
simulations were parameterized by the sonic Mach number, $\mach=v/c_s$, the r.m.s.
velocity relative to the speed of sound, $c_s$; and the \alf\ Mach number,
$\alfmach=v/c_A$, the
r.m.s velocity relative to the speed of magnetic waves, $c_A$.  These are seen
in the legend of the figure, along with the powerlaw exponent of the line.  Two things are
of note.  First, the slopes are all steeper than that found by Planck.  And
second, most notably the bottom red line, is the poor linear quality of the
resulting spectra.  These shortcomings are the result of inadequate resolution,
as discussed in Section \ref{sec.turbulence}.


\subsubsection{Simulations: \nameCMB}
\label{subsec.cmb_sims}

Our previous suite of simulations (Stalpes et al 2022, in prep) was run at a
resolution of  $512^3$, which is insufficient to match the behavior of the sky.
The current simulations will double the size of the inertial range by doubling
the resolution to $1024^3$, and target parameter space suggested by that
parameter suite.  In this case more resolution is better without
bound, as the sky we are trying to match has an astronomical separation of scales. However, doubling
the inertial range increases the cost by a factor of at least 16, so while
higher resolution is possible, is is the expense is not warranted at this time.

The turbulence in our simulations is created by way of large scale forcing. Energy is added to the gas at every time step at the largest scale, in a
manner that keeps the injection rate constant \citep{MacLow99, Schmidt09}.  This is
continued for 5 dynamical times, $t_{dyn}=L/V$, where $L$ is the size of the
pattern and $V$ is the r.m.s. Mach number of the flow.  The first $2 t_{dyn}$
are used to establish a steady state, as the chaos of the onset of turbulence
causes the propertis of the gas to fluctuate wildly at the beginning.   We then run for an additional $3 t_{dyn}$ to build statistics, and
average the power spectra over this window.  This gives us three statistically
independent spectra to average.  Again, more is better, three is the minimum to
achieve a converged spectrum.

Our results from Stalpes et al (2022) indicated that an r.m.s. Mach number of 5
is necessary to reproduce the observed spectra.  We will run a suite of four
simulations with (\mach, \alfmach) = (1,1),(1,5),(5,1),(5,5).  This will allow
us to bracket the ranges of parameters the ISM experiences with a minimum of
simulations.
