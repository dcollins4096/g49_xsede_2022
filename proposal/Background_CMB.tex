The cosmic microwave background (CMB) is the light leftover from the creation of
the universe.  It has taught us a considerable amount about the structure,
history, and future of the
universe.  To learn more from it, we must understand its polarization.  To see
the polarization of the CMB, we must first understand the polarization of the
interstellar medium (ISM), which is much brighter and in the way.  This project
will perform simulations of driven magnetized turbulence, and compare the
filamentary structure found with an analytic model of the CMB polarization
developed by \citet{Huffenberger20}.  We will
motivate this project in Section \ref{subsec.cmb_motivate}, and describe the
simulations we will perform in Section \ref{subsec.cmb_sims}

\subsubsection{Motivation: \nameCMB}
\label{subsec.cmb_motivate}

When the universe was very young, it was very small, and also very hot.  So hot
that everything was ionized.  Sound waves left from the big bang travel through
the universe, causing temperature fluctuations.  Once the universe cooled
enough, the electrons and protons combined to make the first Hydrogen.  After
this time, photons can travel great distances without running into an electron.
These photons are the CMB.  It is an extremely uniform 2.7K black body.  Small
($\mu K$) fluctuations in this temperature have been studied extensively by
satellites such as Planck \citep{Planck2018VI20}. These fluctuations have answered many
questions about the content and eventual fate of the universe.  But there are
still open questions.

Why is the CMB a single temperature?  The
universe is very large, and one side of the observable universe has never been
in contact with the other.  Given the expansion of the universe, one would
expect significant fluctuations in temperature on the angular size of the full moon on the sky.
One possible answer is an extremely rapid
\emph{inflation} of the universe, where the universe expands from the size of a
proton to the size of the solar system within the first $10^{-16}$s.  This is
different from the \emph{expansion} of the universe, which has been well
established and a much more leisurely pace.  Such a
rapid event would allow different patches of the universe to be at the same
temperature very early on.  
It is also extraordinarily violent, would leave a sea of gravitational waves.  These gravitational
waves, being quadrupolar in nature, imprint a polarization on the CMB. To detect
this polarization is to witness the violent birth of the Universe.

Unfortunately (for observing the CMB) the Galaxy we live in is filled with dust,
which gives a polarized signal in the same frequency range as the polarized CMB.
This dust, which includes iron and magnesium, lines up perpendicular to the magnetic
field in the galaxy, not unlike iron filings around a bar magnet.  These aligned
grains radiate polarized thermal radiation in the microwave and infrared.  This
polarized signal is much brighter than the polarization in the CMB, so must be
removed.  In order to remove it, we must understand the statistical properties
of the magnetic field in the Galaxy.

The Planck satellite \citep{PlanckXIX15} measured the galactic magnetic field, and the result can be
seen in Figure \ref{fig.planck}.  Statistically, the polarization is described best
by the quantities $E$ and $B$.  The $E$-mode is the amplitude of the polarized
signal with polarization direction
that is either parallel to or perpendicular to filamentary structures.   The
$B$-mode
describes polarization at oblique angles.  It is found that both structures are
distributed over all scales in a power-law fashion, with $E \propto k^{-2.35}$,
where $k$ is wavenumber on the sky. $B$ has a similar exponent but half the
amplitude.  

Our group has had success in reproducing similar polarized statistics in two
settings; simulations of driven turbulence, and an analytic model based on
magnetized filaments.
In Stalpes et al 2022 (in prep)  we demonstrated that MHD turbulence can reproduce
similar exponents, with the value of the exponent and amplitude depending on
velocity and magnetic field strength in the turbulence.  In
\citet{Huffenberger20}, we developed a model of the ISM polarization based on
magnetized filaments.   In this model, an ensemble of filaments is parametrized
by their length, width, and the mean magnetic field angle to the filament.  This
ensemble, with the right parameters, can give a polarization signal that matches
that of the sky.
In the proposed simulations, we will combine these two approaches.  We will
perform a series of driven turbulent boxes, as described in
Section \ref{subsec.turb_sims}, but this
time with magnetic fields.  We will then use the filament finding tool DISPERSE
\citep{Sousbie11} to extract filamentary structure, and measure how well our
analytic model reproduces the filament and polarization properties of the
turbulent boxes.

The study done in Stalpes et al (2022, in prep) was a parameter sweep at
moderate resolution ($512^3$).  These new simulations will be larger ($1024^3$)
and target Mach numbers that are higher than the previous family of simulations.

\subsubsection{Simulations: \nameCMB}
\label{subsec.cmb_sims}

Our previous simulations have indicated that, as far as the foregrounds are
concerned, the ISM is most likely supersonic and \sa.  The primary parameters
that dictate the behavior of the turbulence is the sonic Mach number, \mach,
and the \alf Mach number, \alfmach.  These are the ratio of the r.m.s fluid
speed to the sonic speed and \alf\ speed, respectively.  The \alf\ speed is the
speed a disturbance travels along a magnetic field in a plasma.  As the
character of the turbulence changes as both \mach\ and \alfmach\ are increased,
we will perform a minimal 4 step parameter search, with \mach and \alfmach equal
to 1 and 5.  Thus, (\mach, \alfmach) = (1,1), (1,5), (5,1) and (5,5) alternately
varying large and small field and velocity.  These will use the MHD and driving
modules in Enzo.  We will increase the resolution beyond that of our preliminary
runs, to $1024^3$.  We will drive the turbulence for 5 dynamical times, where a
dynamical time is the time for a typical driving pattern to cross the box.  As
the driving pattern is 1/2 the box, $T_{dyn}=0.5/\mach$.  Driving for a number
of dynamical times is important to develop statistically relaxed turbulence, as
well as providing statistics for averaging.  
